\documentclass[12pt,a4paper]{article}
\usepackage[paper=a4paper,margin=2.5cm]{geometry}
\usepackage[utf8]{inputenc}
\usepackage{listings}
\usepackage{xcolor}
\usepackage{float}

\lstset{
    basicstyle=\ttfamily\small,
    breaklines=true,
    columns=fullflexible,
    keepspaces=true,
    frame=single
}

\usepackage{amsmath}
\usepackage{graphicx}
\usepackage{hyperref}
\usepackage[english]{babel}
\usepackage[
    backend=bibtex,
    style=ieee,
    citestyle=ieee
]{biblatex}
\addbibresource{references.bib}
\defbibheading{bibliography}[\refname]{}

\setlength{\parindent}{0pt}

% No extra spacing between lines
\linespread{1.15}

% Simple cover page
\newcommand{\coverpage}{
    \begin{titlepage}
        \centering
        {\Huge \bfseries HPC Project\par}
        \vspace{0.7cm}
        {\large High Performance Computing\par}
        \vfill
        {\large\textbf{Authors:}}\par
        {\large HABAGNO Antonin\par
        Léo GUERIN\par}
        \vspace{2cm}
        {\large\textbf{Date:}}\par
        {\large\today}
    \end{titlepage}
}

\begin{document}

\coverpage

%    \begin{abstract}
%        Your abstract.
%    \end{abstract}

\section{Exercice 1}

\newpage

\section{Exercice 2}

\subsection{Matrix-vector product}

\subsubsection{Introduction}

The objective of this exercice is to analyze the performance characteristics of Matrix-Vector Multiplication specifically for tridiagonal matrices. We will implement a sequential baseline, parallelize it using OpenMP and MPI, and analyze the impact of the number of threads and processes on the execution time.

A tridiagonal matrix is a sparse matrix where non-zero elements exist only on the main diagonal, the super-diagonal, and the sub-diagonal.


\subsubsection{Initial Implementation (Naive Storage)}

In the initial phase, we utilized a standard dense matrix representation (int**) to store the tridiagonal matrix. Although the matrix is sparse, memory was allocated for all $ N \times N $ elements, with zeros filled explicitly.


\begin{lstlisting}[language=C]
int** random_tridiagonal_matrix(int n) {
    // 1. Allocation of N pointers (Rows)
    int** matrix = malloc(n * sizeof(int*));
    for (int i = 0; i < n; i++) {
        // 2. Allocation of N integers per row
        matrix[i] = malloc(n * sizeof(int));
    }

    // 3. Explicit initialization of zeros (O(N^2) operation)
    for (int i = 0; i < n; i++) {
        for (int j = 0; j < n; j++) {
            matrix[i][j] = 0;
        }
    }

    // 4. Filling the 3 diagonals
    for (int i = 0; i < n; i++) {
        if (i > 0) matrix[i][i-1] = (rand() % 201) - 100; // Lower
        matrix[i][i] = (rand() % 201) - 100;              // Main
        if (i < n-1) matrix[i][i+1] = (rand() % 201) - 100; // Upper
    }
    return matrix;
}
\end{lstlisting}  

\vspace{0.5cm}

While easy to implement, this approach has a space complexity of $ O(N^2) $. For a tridiagonal matrix, only $ 3N-2 $ elements are significant. Allocating $ N^2 $ space leads to massive memory wastage and limits the maximum achievable $ N $ before RAM saturation.


Despite the naive storage, the multiplication algorithm was optimized to respect the sparsity pattern. Instead of performing a full dot product (row $\times$ column) which would take $ O(N^2) $ operations, we restricted the inner loop to strictly calculate the non-zero diagonals.

\begin{lstlisting}[language=C]
for (int i = 0; i < n; i++) {
    int sum = 0;
    // Optimization: Only access potentially non-zero elements
    if (i > 0) {
        sum += matrix[i][i-1] * vec[i-1]; // Sub-diagonal
    }
    sum += matrix[i][i] * vec[i];         // Main diagonal
    if (i < n-1) {
        sum += matrix[i][i+1] * vec[i+1]; // Super-diagonal
    }
    result[i] = sum;
}
\end{lstlisting}

\vspace{0.5cm}

This reduces the computational complexity to $ O(N) $ instead of $ O(N^2) $.


\subsubsection{OpenMP Parallelization}

We parallelized the optimized loop using pragma omp parallel for.

\begin{lstlisting}[language=C]
#pragma omp parallel for
for (int i = 0; i < n; i++) {
    int sum = 0;
    if (i > 0) sum += matrix[i][i-1] * vec[i-1];
    sum += matrix[i][i] * vec[i];
    if (i < n-1) sum += matrix[i][i+1] * vec[i+1];
    result[i] = sum;
}
\end{lstlisting}


\subsubsection{Experimental Results $ (N=10,000) $}

\vspace{0.5cm}

\begin{tabular}{|c|c|c|c|c|}
    \hline
    \textbf{Implementation} & \textbf{Threads} & \textbf{Time (s)} & \textbf{Observation} \\
    \hline
    Sequential & 1 & 0.000129 & Baseline \\
    OpenMP & 1 & 0.001288 & 10x Slower \\
    OpenMP & 2 & 0.000309 & Slower than Seq \\
    OpenMP & 4 & 0.000645 & Slower than Seq \\
    OpenMP & 6 & 0.000190 & Slower than Seq \\
    OpenMP & 8 & 0.000246 & Slower than seq \\
    \hline
\end{tabular}


\begin{figure}[H]
    \centering
    \includegraphics[width=0.8\textwidth]{img/matrix_vector_naive.png}
    \caption{Times in function of the number of threads / procs}
    \label{fig:omp_mpi}
\end{figure}

\vspace{0.5cm}


The OpenMP implementation failed to provide speedup for this dataset size. This can be attributed to two factors:

\begin{enumerate}
    \item For N=10,000, the total workload is roughly $ 30,000 $ arithmetic operations. Modern CPUs can execute this in microseconds. The overhead introduced by OpenMP (creating threads, scheduling chunks, and synchronization barriers) is significantly higher than the computation time itself. As seen in the table, the 1-thread OpenMP run includes this overhead without any parallel gain, resulting in a 10x slowdown.
    \item The $int**$ structure stores rows in non-contiguous memory locations. When multiple threads try to access matrix[$i$] and vec, they compete for memory bandwidth. Furthermore, since the computation is extremely light, the CPU spends more time fetching data from RAM than calculating.
\end{enumerate}

\subsubsection{Limitations and Proposed Solution}
To observe true parallel speedup, we must increase $N$ significantly (e.g., $N>10^7$). However, the current $int**$ storage makes this impossible : 
\begin{enumerate}
    \item For $N=1 000 000$, an $N\times N$ integer matrix requires $10^{12}$ integers $\approx 4$ TB of RAM.
    \item This causes the program to crash or swap heavily before we can reach a problem size where OpenMP becomes efficient.
\end{enumerate}


To enable large-scale testing and improve cache locality, we must abandon the int** representation. \newline
We will transition to a Compressed Tridiagonal Storage format using three 1D arrays:
\begin{enumerate}
    \item lower (size $N-1$)
    \item main (size $N$)
    \item upper (size $N-1$)
\end{enumerate}

This method reduces space from $O(N^2)$ to $O(N)$. This allows testing $N=100,000,000$ on standard hardware.


\subsubsection{OpenMP Parallelization with Optimized Storage}

With the $ O(N) $ storage, we were able to increase the problem size to $ N = 100 000 000 $. We applied OpenMP parallelism to the optimized loop.

\begin{lstlisting}[language=C]
#pragma omp parallel for
for (int i = 1; i < n - 1; i++) {
  result[i] = matrix->lower[i - 1] * vec[i - 1] + 
              matrix->main[i] * vec[i] +
              matrix->upper[i] * vec[i + 1];
}
\end{lstlisting}


\subsubsection{MPI Parallelization with Optimized Storage}

We implemented MPI parallelization to distribute the workload across multiple processes. The input vectors are decomposed into chunks, and each process is responsible for calculating a specific range of indices. \newline 

A critical challenge in decomposing Tridiagonal Matrix Multiplication is the dependency on neighbors:

\begin{equation}
    y[i] = L[i-1] \times x[i-1] + D[i] \times x[i] + U[i] \times x[i+1]
\end{equation}

To calculate index $i$, a process needs $x[i-1]$ and $x[i+1]$. 

\begin{lstlisting}[language=C]
// Exchange with LEFT neighbor (Rank - 1) to get x[start-1]
if (rank > 0) {
  MPI_Sendrecv(&local_vec[0], 1, MPI_INT, rank - 1, 0, 
               &ghost_left, 1, MPI_INT, rank - 1, 0, MPI_COMM_WORLD, &status);
}

// Exchange with RIGHT neighbor (Rank + 1) to get x[end+1]
if (rank < size - 1) {
  MPI_Sendrecv(&local_vec[local_n - 1], 1, MPI_INT, rank + 1, 0, 
               &ghost_right, 1, MPI_INT, rank + 1, 0, MPI_COMM_WORLD, &status);
}
\end{lstlisting}


\subsubsection{Experimental Results with optimized storage $ (N=100,000,000) $}



\begin{tabular}{|c|c|c|c|c|}
    \hline
    \textbf{Implementation} & \textbf{Threads} & \textbf{Time (s)} & \textbf{Speedup} \\
    \hline
    Sequential & 1 & 0.380672 & Baseline \\
    OpenMP & 2 & 0.221757 & 1.7x Speedup \\
    OpenMP & 4 & 0.119725 & 3.2x Speedup \\
    OpenMP & 6 & 0.080580 & 4.7x Speedup \\
    OpenMP & 8 & 0.078778 & 4.8x Speedup \\
    MPI & 2 & 0.396344 & 0.9x Speedup \\
    MPI & 4 & 0.279268 & 1.4x Speedup \\
    MPI & 6 & 0.242146 & 1.6x Speedup \\
    MPI & 8 & 0.275798 & 1.4x Speedup \\
    \hline
\end{tabular}

\begin{figure}[H]
    \centering
    \includegraphics[width=0.8\textwidth]{img/matrix_vector_opti.png}
    \caption{Times in function of the number of threads / procs}
    \label{fig:omp_mpi}
\end{figure}


Contrary to OpenMP, the MPI implementation shows poor performance on a single shared-memory machine.

\begin{enumerate}
\item Communication Overhead: The operations $ MPI\_Scatterv $ and $ MPI\_Gatherv $ require Rank 0 to serialize and transmit the entire dataset (approx 1.2 GB for N=108) to other processes. In a shared-memory context, this involves expensive memory-to-memory copies that are slower than direct pointer access used in OpenMP.
\item Computation-to-Communication Ratio: The matrix-vector product is an O(N) operation with very few arithmetic operations per byte loaded. The time saved by parallelizing the calculation is overshadowed by the time spent distributing the data.
\end{enumerate}

\subsubsection{Matrix-vector multiplication conclusion}

This study demonstrates that Data Structures are the primary driver of performance in sparse linear algebra. Switching from int** to Compressed Storage was necessary to even run the benchmark.

\begin{enumerate}
\item OpenMP proved highly effective, offering a ~5x speedup on 8 threads, limited only by the physical memory bandwidth of the machine.
\item MPI, while correctly implemented with Halo Exchanges, is not suitable for this specific workload on a single node due to the high cost of data distribution relative to the low computational intensity of the kernel. MPI would only become beneficial if the problem size N exceeded the RAM of a single machine, requiring distribution across a cluster.
\end{enumerate}



\subsection{Matrix Power}

\subsubsection{Introduction}
In this section, we extend our analysis to matrix exponentiation, specifically computing $A^2$ and $A^3$ for a tridiagonal matrix $A$.
Unlike matrix-vector multiplication, squaring a sparse matrix increases its bandwidth.
\begin{itemize}
    \item If $A$ is tridiagonal (bandwidth 1), $A^2$ is pentadiagonal (bandwidth 2).
    \item $A^3$ is heptadiagonal (bandwidth 3).
\end{itemize}
To maintain $O(N)$ memory complexity, we implemented custom data structures to store only the non-zero diagonals.

\subsubsection{Sequential Implementation}
We implemented specific functions to compute the diagonals of the resulting matrices directly.
For $A^2$, the element $(A^2)_{ij}$ is given by $\sum_k A_{ik} A_{kj}$. Due to the sparsity of $A$, only a few terms in this sum are non-zero.

\begin{lstlisting}[language=C]
// Example: Computing the main diagonal of A^2
int val = M[i] * M[i];
if (i > 0) val += L[i - 1] * U[i - 1];
if (i < n - 1) val += U[i] * L[i];
R->main[i] = val;
\end{lstlisting}

For $A^3$, we compute $A^3 = A \times A^2$. This involves multiplying a tridiagonal matrix by a pentadiagonal matrix.

\subsubsection{OpenMP Parallelization}
Since the computation of each row of the result matrix is independent, we can trivially parallelize the outer loops using OpenMP.
We used \texttt{\#pragma omp parallel for} to distribute the rows among threads.

\begin{lstlisting}[language=C]
#pragma omp parallel for
for (int i = 0; i < n; i++) {
    // Compute diagonals for row i
    // ...
}
\end{lstlisting}

\subsubsection{Experimental Results ($N=100,000,000$)}

We measured the execution time for computing $A^2$ and $A^3$ with varying numbers of OpenMP threads.

\textbf{Results for $A^2$:}

\begin{tabular}{|c|c|c|c|}
    \hline
    \textbf{Implementation} & \textbf{Threads} & \textbf{Time (s)} & \textbf{Speedup} \\
    \hline
    Sequential & 1 & 1.239 & Baseline \\
    OpenMP & 2 & 0.747 & 1.65x \\
    OpenMP & 4 & 0.383 & 3.23x \\
    OpenMP & 6 & 0.277 & 4.47x \\
    OpenMP & 8 & 0.290 & 4.27x \\
    \hline
\end{tabular}

\begin{figure}[H]
    \centering
    \includegraphics[width=0.8\textwidth]{img/matrix_power2.png}
    \caption{Execution time for $A^2$ vs number of threads}
    \label{fig:matrix_power2}
\end{figure}

\textbf{Results for $A^3$:}

\begin{tabular}{|c|c|c|c|}
    \hline
    \textbf{Implementation} & \textbf{Threads} & \textbf{Time (s)} & \textbf{Speedup} \\
    \hline
    Sequential & 1 & 8.143 & Baseline \\
    OpenMP & 2 & 3.939 & 2.06x \\
    OpenMP & 4 & 2.028 & 4.01x \\
    OpenMP & 6 & 1.469 & 5.54x \\
    OpenMP & 8 & 1.272 & 6.40x \\
    \hline
\end{tabular}

\begin{figure}[H]
    \centering
    \includegraphics[width=0.8\textwidth]{img/matrix_power3.png}
    \caption{Execution time for $A^3$ vs number of threads}
    \label{fig:matrix_power3}
\end{figure}

\subsubsection{Analysis}
For $A^2$, we observe good scalability up to 6 threads. At 8 threads, performance slightly degrades, likely due to memory bandwidth saturation, as the arithmetic intensity of $A^2$ is relatively low.
For $A^3$, which involves more arithmetic operations per element (multiplying tridiagonal by pentadiagonal), the scalability is better, achieving a 6.4x speedup on 8 threads. This suggests that the more computationally intensive the kernel, the better the parallel efficiency on this hardware.

\end{document}